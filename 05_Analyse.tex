% Analyse
% ========

\chapter{Analyse de la thèse de Church-Turing}
\label{sec:analyse_de_la_th_se_de_church_turing}
Dans ce chapitre on va principalement voir ce qu'est un bon formalisme.

\section{Fondement de la thèse}
\label{sub:fondement_de_la_th_se}
La forme originale ne contient que 2 parties :
\begin{enumerate}
	\item Toute fonction calculable par une MT est effectivement calculable
	\item Toute fonction effectivement calculable est effectivement
		calculable par une MT
\end{enumerate}
La partie 1 est démontrée et la 2 est supposée vraie.
On la suppose vraie, parce qu'on a des évidences heuristiques (il y a eu beaucoup
d'essais pour trouver une fonction qui ne respectait pas la thèse) et parce qu'on a montré
l'équivalence entre tous les formalismes créés à ce jour (on a montré que toute
machine constructible par la mécanique de Newton ne calcule que des fonctions
calculables).
% subsection fondement_de_la_th_se (end)

\section{Formalismes de la calculabilité}
\label{sub:formalismes_de_la_calculabilit_}
On va se poser la question de ce qui fait un bon formalisme de la
calculabilité. Il faut un formalisme qui vérifie les fondements de la
thèse. Plus précisément, on va étudier des caractéristiques (des propriétés)
nécessaires et suffisantes pour avoir un bon modèle de la calculabilité.

\paragraph{} Dans ce chapitre, on va considérer un formalisme de la
calculabilité D.

\paragraph{Caractéristiques}
\begin{description}
	\item[SD] Soundness\footnote{Cohérence} des descriptions
	\item[CD] Complétude des descriptions
	\item[SA] Soundness algorithmique
	\item[CA] Complétude algorithmique
	\item[U] Description universelle
	\item[S] Propriété S-m-n affaiblie
\end{description}


\begin{myrem}
	\textbf{D} correspond à description, c'est-à-dire une description de fonction.\\
	\textbf{A} correspond à algorithme, c'est-à-dire à une description
	exécutable.
\end{myrem}

\begin{myrem}
	\textbf{Soundness} signifie que, si on a une description dans notre modèle D alors
	celle-ci est cohérente, correspond bien à une fonction
	calculable.\\
	\textbf{Complétude} signifie que notre modèle est complet, qu'il
	n'existe pas de fonction calculable qui ne le soit pas dans notre
	modèle.
\end{myrem}

\paragraph{Caractérisiques plus en détails}
\begin{description}
	\item[SD]  Toute fonction D-calculable est calculable (première partie
		de la thèse de Turing)
	\item[CD]  Toute fonction calculable est D-calculable (deuxième partie
		de la thèse de Turing)
	\item[SA]  L'interpréteur de D est calculable (on peut exécuter une
		description de programme de D)
	\item[CA]  Il existe un
		compilateur qui étant donné un programme $p$ dans un formalisme
		respectant SA produit une description de programme $d \in D$ tel que
		$p$ et $d$ calculent la même fonction.
	\item[U]  L'interpréteur de D est D-calculable (sinon on sait par
		Hoare-Allison que ce n'est pas un formalisme complet).
	\item[S] Il existe un transformateur de programme calculable, qui,
		recevant comme entrée un programme $d \in$ D à 2 arguments et une valeur
		$x$, fournit comme résultat un programme $d'$ tel que $d'(y)$ calcule
		la même fonction que $d(x,y)$.
		% checker si d' dans D!!!
\end{description}
Un bon formalisme de la calculabilité possède toutes ces propriétés. En pratique, il
suffit d'en montrer certaines, car certaines propriétés en entraînent d'autres.

\begin{myprop}
	SA $\Rightarrow$ SD, car si on peut trouver une description exécutable,
	celle-ci existe.
\end{myprop}

\begin{myprop}
	CA $\Rightarrow$ CD, car si on a un compilateur qui compile un programme
	de P en une description dans D, alors un programme de P est calculable dans D.
\end{myprop}

\begin{myprop}
	SD et U $\Rightarrow$ SA car si on sait qu'il existe une description D calculant
	une fonction et
qu'on a un interpréteur de D D-calculable, alors la description est exécutable.
\end{myprop}

\begin{myprop}
	CD et S $\Rightarrow$ CA, considérons n'importe quel autre formalisme
Q qui a la propriété SA (ce qui signifie que sont interpréteur, $interp_Q(n,x)$ est calculable).
Donc par la propriété CD, il existe un programme $p\in P$ tel que
$\phi_p=\phi_{interp_Q}(n,x)$. Ensuite par la propriété S, il existe T, un
transformateur de programme tel que : $\phi_{interp_Q}(n,x)=\phi_{T(n)}(x)$.
On peut donc voir T comme un programme qui compile/transforme une description $q\in Q$ en une
description $p\in P$. Ce qui implique qu'on a bien la propriété CA.

\end{myprop}

Un bon formalisme doit donc posséder soit SA et CA ou soit SD, CD, U et S ou
soit SA, CD et S ou encore CA, SD et U, car
\begin{itemize}
	\item SA et CA $\iff$ SD, CD, U et S
	\item SA, CD et S $\iff$ SD, CA et U
\end{itemize}
% su section formalismes_de_la_calculabilité_ (end)

\section{Techniques de preuve}
\label{sub:techniques_de_preuve}

Pour prouver qu'un problème est non calculable, on peut utiliser :
\begin{itemize}
	\item Le théorème de Rice
	\item La démonstration directe de la non-calculabilité par
		diagonalisation ou par preuve par l'absurde.
		\textbf{A priori le plus dur !} C'est plus pratique de réutiliser les problèmes pour
		lesquels on a déjà montré la non-calculabilité.
	\item La méthode de réduction
\end{itemize}
% subsection techniques_de_preuve (end)

\section{Aspects non couverts par la calculabilité\\ (Section pas très
	importante pour l'examen)}
\label{sub:aspects_non_couvert_par_la_calculabilit_}
La calculabilité se limite au calcul de fonctions. Or certains problèmes de la vie
de tous les jours ne correspondent pas à une fonction.

\begin{myexem}
	Système d'exploitation
\end{myexem}

\begin{myexem}
	Système de réservation aérienne
\end{myexem}

\begin{myexem}
	Certains problèmes utilisent des caractéristiques de l'environnement comme
	des données en provenance de sonde.
\end{myexem}
% subsection aspects_non_couvert_par_la_calculabilit_ (end)

\section{Au-delà de la calculabilité}
\label{sub:au_del_de_la_calculabilit_}
Est-il possible d'imaginer un modèle plus puissant que les modèles qu'on a
aujourd'hui? Est-ce que les humains sont ``plus puissants'' que les machines dans
le sens où ils pourraient calculer des fonctions non calculables?

Ce sont des questions qui font débat. Certain ne veulent même pas
poser la deuxième question c'est-à-dire est-ce que H-calculable = T-calculable.

\begin{mydef}[H-calculable]
	Une fonction est H-calculable si un être humain
	est capable de calculer cette fonction.
	% tu confonds avec la thèse CT ``procédés publics'' sinon
\end{mydef}

\paragraph{} À la question ``les machines pensent-elles?'', Turing a proposé un
raisonnement :
\begin{itemize}
	\item \textbf{Hypothèse :} les machines pensent.
	\item Si on ne peut réfuter l'hypothèse alors celle-ci est vraie
	\item Envisager toutes les objections (théologique, émotions,
		mathématiques...) et toutes les réfuter.
\end{itemize}

% subsection au_del_de_la_calculabilit_ (end)
% section analyse_de_la_t_se_de_chu ch_Turin (end)


